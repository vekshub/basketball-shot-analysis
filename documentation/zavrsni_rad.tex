\documentclass[zavrsnirad]{fer}
% Dodaj opciju upload za generiranje konačne verzije koja se učitava na FERWeb
% Add the option upload to generate the final version which is uploaded to FERWeb


\usepackage{blindtext}


%--- PODACI O RADU / THESIS INFORMATION ----------------------------------------

% Naslov na engleskom jeziku / Title in English
\title{Analysis of basketball training through a mobile application}

% Naslov na hrvatskom jeziku / Title in Croatian
\naslov{Analiza treninga košarke putem mobilne aplikacije}

% Broj rada / Thesis number
\brojrada{1936}

% Autor / Author
\author{Vedran Kumanović}

% Mentor 
\mentor{izv. prof. dr. sc.\@ Matko Orsag}

% Datum rada na engleskom jeziku / Date in English
\date{June, 2025}

% Datum rada na hrvatskom jeziku / Date in Croatian
\datum{lipanj, 2025.}

%-------------------------------------------------------------------------------


\begin{document}


% Naslovnica se automatski generira / Titlepage is automatically generated
\maketitle


%--- ZADATAK / THESIS ASSIGNMENT -----------------------------------------------

% Zadatak se ubacuje iz vanjske datoteke / Thesis assignment is included from external file
% Upiši ime PDF datoteke preuzete s FERWeb-a / Enter the filename of the PDF downloaded from FERWeb
\zadatak{hr_1191247427_73.pdf}


%--- ZAHVALE / ACKNOWLEDGMENT --------------------------------------------------

\begin{zahvale}
  Hvala svima koji su mi pomogli u radu na ovom završnom radu.
  
\end{zahvale}


% Odovud započinje numeriranje stranica / Page numbering starts from here
\mainmatter


% Sadržaj se automatski generira / Table of contents is automatically generated
\tableofcontents


%--- UVOD / INTRODUCTION -------------------------------------------------------
\chapter{Uvod}
\label{pog:uvod}

Neki od radova koje ćemo citirati su \cite{6248073,6247753,ghiglia_pritt_phase_unwrapping,hartley2003multiple,4250461,123DCatch}.
Trebaju nam samo radi testiranja kako izgleda referenciranje rada s konferencije, rada iz časopisa, knjige i Internetske stranice.

\begin{figure}[htb]
  \centering
  \includegraphics[width=0.38\linewidth]{Figures/lenovo_yoga_tab3_pro_front.png} 
  \caption{Moja prva slika}
  \label{slk:prvaslika}
\end{figure}

Referenciramo se na sliku \ref{slk:prvaslika} u sredini rečenice, zatim prije zareza \ref{slk:prvaslika}, te zatim na kraju rečenice \ref{slk:prvaslika}.
Upravo smo testirali radi li naredba \verb|\ref| ispravno u slučaju kada nakon nje slijedi točka.

Sada slijedi jedna jednadžba:
\begin{equation}
  \label{jed:prvajednadzba}
  \int_{-\infty}^{+\infty}f(t)\,dt=F(\omega)
\end{equation}
Jednadžba \eqref{jed:prvajednadzba} je moja prva jednadžba koja defnira par $f(t)\ufrek F(\omega)$ ili $F(\omega)\uvrem f(t)$.

%---KORIŠTENE TEHNOLOGIJE I ALATI / USED TECHNOLOGIES AND TOOLS -----------------------------
\chapter{Korištene tehnologije i alati}
\label{pog:korištene_tehnologije_i_alati}


%---RAZVOJ SUSTAVA ZA DETEKCIJU I ANALIZU KOŠARKAŠKOG ŠUTA / DEVELOPMENT OF A SYSTEM FOR DETECTING AND ANALYZING BASKETBALL SHOTS
\chapter{Razvoj sustava za detekciju i analizu košarkaškog šuta}
\label{pog:razvoj_sustava_za_detekciju_i_analizu_kosarkaskog_suta}

\section{Prikupljanje i obrada podataka}
\label{pog:prikupljanje_i_obrada_podataka}

\section{Detekcija objekata korištenjem YOLO modela}
\label{pog:detekcija_objekata_koristenjem_yolo_modela}


\section{Prepoznavanje pokušaja šuta i pogođenih koševa}
\label{pog:prepoznavanje_pokusaja_suta_i_pogodenih_koseva}

\section{Rekonstrukcija 2D i 3D putanje lopte}
\label{pog:rekonstrukcija_2d_i_3d_putanje_lopte}

\section{Određivanje kuta izbačaja i brzine šuta}
\label{pog:odredivanje_kuta_izbacaja_i_brzine_suta}


\section{Mogućnosti prijenosa sustava na mobilne uređaje}
\label{pog:mogucnosti_prijenosa_sustava_na_mobilne_uredaje}


%---REZULTATI I RASPRAVA / RESULTS AND DISCUSSION -----------------------------
\chapter{Rezultati i rasprava}
\label{pog:rezultati_i_rasprava}

\section{Primjeri detekcije i analize šuteva}
\label{pog:primjeri_detekcije_i_analize_suteva}
\section{Evaluacija točnosti detekcije i analize}
\label{pog:evaluacija_tocnosti_detekcije_i_analize}
\subsection{Eksperimenti u kojima se razvijeni postupak dobro ponaša}
\label{pog:eksperimenti_u_kojima_se_razvijeni_postupak_dobro_ponasa}
\subsection{Eksperimenti u kojima postupak nalazi objekte koje nismo tražili (false positive error)}
\label{pog:eksperimenti_u_kojima_postupak_nalazi_objekte_koje_nismo_trazili}

\subsection{Eksperimenti u kojima postupak ne pronalazi tražene objekte (false negative error)}
\label{pog:eksperimenti_u_kojima_postupak_ne_pronalazi_trazene_objekte}


%--- ZAKLJUČAK / CONCLUSION ----------------------------------------------------
\chapter{Zaključak}
\label{pog:zakljucak}

\blindtext


%--- LITERATURA / REFERENCES ---------------------------------------------------

% Literatura se automatski generira iz zadane .bib datoteke / References are automatically generated from the supplied .bib file
% Upiši ime BibTeX datoteke bez .bib nastavka / Enter the name of the BibTeX file without .bib extension
\bibliography{literatura}



%--- SAŽETAK / ABSTRACT --------------------------------------------------------

% Sažetak na hrvatskom
\begin{sazetak}
  Unesite sažetak na hrvatskom.

  \blindtext
\end{sazetak}

\begin{kljucnerijeci}
  prva ključna riječ; druga ključna riječ; treća ključna riječ
\end{kljucnerijeci}


% Abstract in English
\begin{abstract}
  Enter the abstract in English.
  
  \blindtext 
\end{abstract}

\begin{keywords}
  the first keyword; the second keyword; the third keyword
\end{keywords}


%--- PRIVITCI / APPENDIX -------------------------------------------------------

% Sva poglavlja koja slijede će biti označena slovom i riječi privitak / All following chapters will be denoted with an appendix and a letter
\backmatter

\chapter{The Code}

\Blindtext


\end{document}
